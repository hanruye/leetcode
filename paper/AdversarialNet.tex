\documentclass[10pt,letterpaper]{article}
\usepackage{cvpr}
\usepackage{times}
\usepackage{epsfig}
\usepackage{graphicx}
\usepackage{amsmath}
\usepackage{amssymb}
\usepackage{subfigure}

\usepackage[pagebackref=true,breaklinks=true,letterpaper=true,colorlinks,bookmarks=false]{hyperref}

\cvprfinalcopy


\def\cvprPaperID{****} % *** Enter the CVPR Paper ID here
\def\httilde{\mbox{\tt\raisebox{-.5ex}{\symbol{126}}}}

\ifcvprfinal\pagestyle{empty}\fi

\begin{document}
\title{Generative Adversarial Nets}

\author{Ruyue Han\\\\ \today}

\maketitle
%\thispagestyle{empty}

\section{Adversarial nets}
The adersarial nets include two parts: generator G and discriminator D,they are all neural networks and can be changed following your mind.The purpose of G is to create images ,the D is to identify the source of images.In other words,D and G play the folloing two-player minimax game with value function V(G,D).\par
\begin{equation}
\begin{aligned}
\min_{G}\max_{D} V(G,D) &=  
\mathbb{E}_{x \backsim p_{data}(x)} \left[\log{D(x)} \right] 
+ \mathbb{E}_{z \backsim p_{z}(z)}\left[\log{1-D(G(z))} \right]\\
&=\mathbb{E}_{x \backsim p_{data}(x)} \left[\log{D(x)} \right] 
+ \mathbb{E}_{x \backsim p_{g}(x)}\left[\log{1-D(x)} \right]
\end{aligned}
\end{equation}
where $p_{data}(x)$ is the real data's distribution, and $p_{z}(z)$ is the noise's distribution which can be changed, $p_{g}(x)$ is generator’s distribution.\par
The final purpose is that G can create realistic image and D can't identify where the image is coming from,which means the generator distribution $p_{g}$ is closed to real data distribution $P_{data}$.The training steps can be see as Fig.\ref{fig:firstpicture1}
\begin{figure*}[htb]
\centering
\includegraphics[width=6.50in,height=1.80in]{9-2.png}
\caption{Generative adversarial nets are trained by simultaneously updating the discriminative distribution (D, blue, dashed line) so that it discriminates between samples from the data generating distribution (black, dotted line) $p_{data}$ from those of the generative distribution $p_{g}$ (G) (green, solid line).}
\label{fig:firstpicture1}
\end{figure*}
\section{Theoretical Results}
\subsection{Global Optimality of $p_{g} = p_{data}$}
Author first consider the optimal discriminator D for any given generator G.
\begin{equation}
\begin{aligned}
V(G,D) &= \mathbb{E}_{x \backsim p_{data}(x)} \left[\log{D(x)} \right]
+ \mathbb{E}_{x \backsim p_{g}(x)}\left[\log{1-D(x)} \right]\\
&= \int_{x} p_{data}(x)\log{(D(x))} dx + \int_{x} p_{g}(x)log(1-D(x))dx\\
&= \int_{x} p_{data}(x)\log{(D(x))} + p_{g}(x)log{(1-D(x))} dx
\end{aligned}
\end{equation}
For any $(a,b)\in{\mathbb{R}^{2}}$,the function $f(x) = alog(x)+blog(1-x),0 < x < 1 $,it's maximum is $\frac{a}{a+b}$.\par
Proof:the f(x) is differentiable function,when $0 < x < 1$,and we know when the derivative function $f^{'}(x) = 0$,the value of f(x) is maxmun. 
\begin{equation}
f^{'}(x) = \frac{a}{x}+\frac{-b}{1-x} = 0\\ 
\end{equation}
so the answer is $\frac{a}{a+b}$.
Now we can get the optimal discriminator D is
\begin{equation}
D^{*}(x) = \frac{p_{data}(x)}{p_{data}(x)+p_{g}(x)}
\end{equation}
and at this time
\begin{equation}
\begin{aligned}
\max_{D}V(G,D) &= V(G,D^{*}) = 
\mathbb{E}_{x \backsim p_{data}(x)} \left[\log{D^{*}(x)} \right] 
+ \mathbb{E}_{x \backsim p_{g}(x)}\left[\log{(1-D^{*}(x))} \right]\\
&= \mathbb{E}_{x \backsim p_{data}(x)} \left[\log{ \frac{p_{data}(x)}{p_{data}(x)+p_{g}(x)}} \right] 
+ \mathbb{E}_{x \backsim p_{g}(x)}\left[\log{\frac{p_{g}(x)}{p_{data}(x)+p_{g}(x)}} \right]\\
&= \dots\\
&= -\log{4} + KL(p_{data}(x)||\frac{p_{data}(x)+p_{g}(x)}{2})+ KL(p_{g}(x)||\frac{p_{data}(x)+p_{g}(x)}{2})\\
&= -\log4 + 2JS(p_{data}(x)||p_{g}(x))
\end{aligned}
\end{equation}
Given different G we got different JS distance,and Our purpose is to get the shortest JS distence.So only and only $p_{data} == p_{g}$,the generative model perfectly replicate the data generating process.\par
In the practice, we update the discriminator by ascending its stochastic gradien and update the generator by descending its stochastic gradient.
\begin{figure*}[htb]
\centering
\includegraphics[width=6.50in,height=4.00in]{9-3.png}
\caption{The algorithm of minmax game.}
\label{fig:firstpicture2}
\end{figure*}
\section{Theoretical Results}


{\small
\bibliographystyle{ieee}
\bibliography{DarkChannelPrior}
}

\end{document}
