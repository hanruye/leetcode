\documentclass[10pt,twocolumn,letterpaper]{article}
\usepackage{cvpr}
\usepackage{times}
\usepackage{epsfig}
\usepackage{graphicx}
\usepackage{amsmath}
\usepackage{amssymb}
\usepackage{subfigure}
\usepackage{float}
\usepackage{tabularx}

\usepackage[pagebackref=true,breaklinks=true,letterpaper=true,colorlinks,bookmarks=false]{hyperref}

\cvprfinalcopy


\def\cvprPaperID{****} % *** Enter the CVPR Paper ID here
\def\httilde{\mbox{\tt\raisebox{-.5ex}{\symbol{126}}}}

\ifcvprfinal\pagestyle{empty}\fi

\begin{document}
\title{Locally Adaptive Color Correction for Underwater Image Dehazing and
Matching}

\author{Ruyue Han\\\\ \today}

\maketitle
%\thispagestyle{empty}

%%%%%%%%% ABSTRACT



\section{Underwater Dehazing and Transmission Estimation}
Author have given the simplified underwater optical model as follows:
\begin{equation}
I(x) = J(x)e^{-\eta d(x)}+B_{\infty}(x)(1-e^{-\eta d(x)}) \label{equ:eqone}
\end{equation}
Author used the method of He et al.~\cite{image2011_1} to dehaze the underwater images.And based on Eq \ref{equ:eqone},author give the transmission as:
\begin{equation}
t(x)=1- \min_{y\in{\Omega (x)}}  \left( \min_{c\in{r,g,b}}I^{c}_{CC}(y)/ {B_{\infty}}^{c}    \right)
\end{equation}


\section{Results and Discussion}
\paragraph{Transmission Qualitative Evaluation}
Author think their  color correction approach is quit good robust and stable compared with many other DCP related methods.And we can see the result from Fig.~\ref{fig:figure1}.
\begin{figure*}[h]
\centering
\includegraphics[width=6.00in,height=3.50in]{6-6.PNG}
\caption{Estimated transmission maps generated by different specialized underwater techniques (DCP [17], MDCP [13], UDCP [20]) and our approach generated with four different reference images.}
\label{fig:figure1}
\end{figure*}
\paragraph{Underwater Dehazing Evaluation}
As we can see from Fig ~\ref{fig:figuretwo} and Table ~\ref{tab:table1}, the method of author is able to yield comparative and even better outputs compared with the other analyzed techiques.
\begin{figure*}[h]
\centering
\includegraphics[width=6.00in,height=5.50in]{6-10.PNG}
\caption{Estimated transmission maps generated by different specialized underwater techniques (DCP [17], MDCP [13], UDCP [20]) and our approach generated with four different reference images.}
\label{fig:figuretwo}
\end{figure*}
\begin{table}[h]
\caption{Average values shown in Fig ~\ref{fig:figuretwo}}
\label{tab:table1}
\begin{tabular}{|c|c|c|c|c|c|}
%	\hline 
%	Dataset & \textit{\textbf{MSMT17}} & \textit{Duke} \cite{zheng2017unlabeled}& %\textit{Market} \cite{zheng2015scalable}& \textit{CUHK03} \cite{li2014deepreid}& %\textit{CUHK01} \cite{li2012human}& \textit{VIPeR} \cite{gray2008viewpoint}& %\textit{PRID} \cite{hirzer2011person}& \textit{CAVIAR} \cite{cheng2011custom}\\
%	\hline
	\hline
	 & \textbf{DCP} & \textbf{MDCP} & \textbf{UDCP} & \textbf{Only CC} & \textbf{Author} \\ 	
	\hline
	\textbf{UCIQUE} & 0.4690 & 0.4927 & 0.5201 & 0.4688 & 0.5691 \\ 
	\hline 
	\textbf{PCQI} & 0.9049 & 0.9081 & 0.8038 & 0.8417 & 0.9634 \\ 
	\hline 
\end{tabular}
\end{table}


%\begin{figure*}
%\centering
%\subfigure[average adversarial loss]{
%\label{Fig.sub.1}
%\includegraphics[width=3.00in,height=2.20in]{5-1.PNG}}
%\subfigure[proposed max adversarial loss]{
%\label{Fig.sub.2}
%\includegraphics[width=3.00in,height=2.20in]{5-2.PNG}}
%\caption{Average adversarial loss versus max adversarial
%loss during the training}
%\label{Fig.lable11}
%\end{figure*}

\paragraph{Underwater Image Matching}
The effects of strong scattering in underwater adds complexity to various computer vision algorithms such as detection and localization.Author prove the utility of their technique for the task of matching images based on local feature points considering the well-known SIFT operator ~\cite{Color2001_4} and get a better result than other technique.

\section{Conclusion}
In this paper author introduce a simple but effective color correction approach for underwater images. Inspired by the DCP ~\cite{Single2009_2} and the simple color transfer approach of ~\cite{Color2001_3},their comprehensive evaluation demonstrates that the strategy can effectively estimate transmission map and remove the haze effect for various underwater scenes. Moreover, author prove the utility of our dehazing method for a fundamental underwater computer vision application: matching images based on local feature points.


{\small
\bibliographystyle{ieee}
\bibliography{underwaterImage}
}

\end{document}
