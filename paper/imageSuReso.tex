\documentclass[10pt,twocolumn,letterpaper]{article}
\usepackage{cvpr}
\usepackage{times}
\usepackage{epsfig}
\usepackage{graphicx}
\usepackage{amsmath}
\usepackage{amssymb}
\usepackage{subfigure}

\usepackage[pagebackref=true,breaklinks=true,letterpaper=true,colorlinks,bookmarks=false]{hyperref}

\cvprfinalcopy


\def\cvprPaperID{****} % *** Enter the CVPR Paper ID here
\def\httilde{\mbox{\tt\raisebox{-.5ex}{\symbol{126}}}}

\ifcvprfinal\pagestyle{empty}\fi

\begin{document}
\title{To learn image super-resolution, use a GAN to
learn how to do image degradation first}

\author{Ruyue Han\\\\ \today}

\maketitle
%\thispagestyle{empty}

%%%%%%%%% ABSTRACT
\begin{abstract}
   This paper is on image and face super-resolution.Author think prior work is not really good for increasing the quality of real-world low-resolution images.So author gives a new way to resolve this problem.This way include two steps,one is training a Hight-to-Low resolution images GAN net using unpaired images.Then trainning a Low-to-High GAN using the images from High-to-Low GAN to transfer low-resolution to high.
   
   Keywords: Image and face super-resolution, Generative Adversarial
Networks, GANs.

\end{abstract}
\section{Introduction}
	Author said that at present,there are a lot of papers focusing on image and face super-resolusion,but most of them using images artificially generated by simple bilinear down-sampling.Author's paper presents one of the very first attempts towards real-world image super-resolution.A few results are shown in Fig.~\ref{fig:resolution}
\begin{figure}[htb]
\centering
\includegraphics[width=3.50in,height=1.50in]{4-1.PNG}
\caption{Super-resolution results produced by our system on real-world low-
resolution faces from Widerface ~\cite{Wider2016_1}. Our method is compared against SRGAN ~\cite{Photo2017_2} and CycleGan ~\cite{Unpaired2017_3}.}
\label{fig:resolution}
\end{figure}

\section{Closely related work}
The author's work based on Convolutional Neural Networks (CNNs).And main approach to super-resolution is to use a fully supervised approach where a low-resolution (LR) image is processed by a network comprising convolutional and upsampling layers in order to produce a high-resolution (HR) image which is then matched against the original HR image using an appropriate loss function.As author said that they inspired by the recent work ~\cite{Super2017_4}.
\section{Method}
	This paper gives a way that can tansfer a LR facial image of size 16 × 16 to 
 a HR image of 64 × 64.The overall architecture, which is end-to-end trainable, is shown in Fig.~\ref{fig:resolution1}.
 
 \begin{figure}[htb]
\centering
\includegraphics[width=3.50in,height=1.50in]{4-2.PNG}
\caption{Overall proposed architecture and training pipeline.}
\label{fig:resolution1}
\end{figure}


%\begin{figure}
%\centering
%\subfigure[SubfigureCaption]{
%\label{Fig.sub.1}
%\includegraphics[width=3.50in,height=2.80in]{code1anwser.eps}}
%\subfigure[SubfigureCaption]{
%\label{Fig.sub.2}
%\includegraphics[width=3.50in,height=2.80in]{code1anwser.eps}}
%\caption{MainfigureCaption}
%\label{Fig.lable}
%\end{figure}


{\small
\bibliographystyle{ieee}
\bibliography{imageSuReso}
}

\end{document}
